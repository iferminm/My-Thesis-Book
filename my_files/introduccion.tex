\chapter*{Introducci\'on}
\label{chap:introduccion}
\addcontentsline{toc}{chapter}{Introducción} 

Internet, la autopista de la información, un enorme grafo con información esperando ser descubierta, conocimiento esperando ser asimilado, servicios esperando ser utilizados, documentos esperando ser leídos, en fin, recursos esperando ser explotados. Ciertamente, para muchos, Internet se ha convertido en un medio primario de entretenimiento, aprendizaje o, incluso, un tópico de investigación. Esta revolución tecnológica ha causado gran impacto en el mundo y ha cambiado, incluso, la manera tradicional de hacer las cosas.

Antes de la aparición de Internet: las empresas debían pagar costosos espacios en radio y TV para poder desplegar su maquinaria publicitaria, dejando a las pequeñas empresas, que no pueden pagar dichos espacios, en desventaja, los políticos debían ir casa por casa hablando con la gente, llevar a cabo apariciones públicas en mítines canales de TV y entrevistas de radio para que su mensaje llegara, los artistas dependías de grandes disqueras para dar a conocer su trabajo y su música y pudiera seguir dando cientos de ejemplos. Todo eso ha cambiado desde la aparición de Internet y, más aún, desde la aparición de lo que Tim O'Reilly llamó la \textit{Web 2.0}, esa red interactiva en la que cualquiera puede generar contenido y la información se encuentra a sólo un click de distancia, ahora es posible realizar campañas publicitarias ``dos punto cero'', desde la comodidad de una habitación u oficina, incluso, dar soporte y apoyo a campañas políticas como la de Barack Obama para la Presidencia de los Estados Unidos en 2008, o la menos exitosa de Diego Arria en Venezuela, compitiendo en 2012 por la candidatura a la Presidencia de la República, esta última se apoyó, principalmente, en las redes sociales como Facebook y Twitter, realizando entrevistas y conferencias a través del servicio de TwitCam para dar a conocer sus propuestas, quedando en cuarto lugar logrando cerca de 135.000 votos.

Ciertamente Internet ha cambiado la manera de hacer investigación, hace 20 años para conseguir información acerca de algún tema, el único recurso disponible era ir a una biblioteca y consultar los libros que allí ofrecen, ahora, para conseguir información acerca del mismo tópico basta con ir a \textit{Wikipedia} para tener una idea general y realizar algunas búsquedas para profundizar un poco más, pero, ¿podemos ver la Internet como una gran base de datos?, una base de datos constituye, en su definición más sencilla, un conjunto de datos relacionados entre sí que tienen un significado para alguien (Cardoso, 2006). Ciertamente, los recursos disponibles en la Internet tienen significado para quien los busca, pero, ¿están relacionados entre sí?, para que estén relacionados, debe haber cierta estructura y orden en los datos, ¿tiene Internet esa estructura y ese orden?, más allá de eso, ¿será posible dar orden y estructura a la Internet para manipularla como una gran base de datos?

La respuesta, afirmativa o negativa, a todas esas interrogantes están en \textit{La Web Semántica} o, como algunos la llaman, \textit{Web 3.0}, que busca dar una estructura flexible a los recursos que hacen vida dentro de la \textit{gran autopista de la información} que constituye la WWW. En la \textit{Web 2.0}, el tema álgido consiste en la manera como se modelan o se presentan los datos (Kúster y Rauseo, 2009), en la \textit{Web 3.0}, el tema será la manera como se describen y se relacionan los datos (recursos) entre sí, haciéndolos relevantes para el usuario en la medida en que su descripción se ajuste a la consulta realizada.

El presente trabajo está estructurado de la siguiente manera:

El Capítulo 1, presenta el planteamiento del problema, los objetivos y la justificación de esta investigación, el Capítulo 2, presenta todo el Marco Teórico que da basamento a este trabajo. Seguidamente, el Capítulo 3, describe la metodología utilizada durante el desarrollo de las actividades necesarias para la realización de este proyecto, el Capítulo 4, describe las actividades realizadas durante el desarrollo. A continuación, el Capítulo 5 presenta los resultados obtenidos como producto de la parte práctica de este trabajo, el Capítulo 6 presenta las Conclusiones y una serie de Recomendaciones para trabajos futuros enmarcados dentro de esta línea de investigación. Finalmente se muestran las Referencias Bibliográficas y los Apéndices.

\end{itemize}
