\begin{thebibliography}{20}

\bibitem{swp} Antoniou, Grigori y Van Harmelen, Frank. (2003). A Semantic Web Primer, Massachusetts: MIT Press.

\bibitem{planeta} Cobo, Cristóbal y Pardo, Hugo. (2007) Planeta 2.0: Inteligencia Colectiva o Medios Fast Food. México DF: Grup de Recerca d'Interaccions Digitals.

\bibitem{exito} Éxito Explorador (Agosto 31, 2010). Estadísticas Mundiales de Internet [Datos en línea] en http://www.exitoexportador.com/stats.htm [Consulta: 2010, Noviembre 28].

\bibitem{gbws} W3C. Guía Breve de la Web Semántica [Documento en línea]. Disponible: http://www.w3c.es/divulgacion/guiasbreves/websemantica [Consulta: 2010, Noviembre 28].

\bibitem{ricardo} Casanova, Ricardo. (2010). El Modelo Web 2.0. Presentación sobre el modelo de negocios orientado a la Web 2.0.

\bibitem{rdfp} W3C. RDF Primmer [Documento en línea]. Disponible: http://www.w3.org/TR/2004/REC-rdf-primer-20040210/ [Consulta: 2010, Noviembre 28].

\bibitem{rdf} W3C. Resource Description Framework (RDF) [Documento en línea]. Disponible: http://www.w3.org/RDF/ [Consulta: 2010, Noviembre 29].

\bibitem{rdffaq} GSI. RDF Schema (RDFS) [Presentación en línea]. Disponible: www.gsi.dit.upm.es/~gfer/ssii/RDFS.pdf [Consulta: 2010, Noviembre 30].

\bibitem{owlfaq} W3C. Preguntas Frecuentes del OWL [Documento en línea]. Disponible: http://www.w3c.es/Traducciones/es/SW/2005/owlfaq [Consulta: 2010, Noviembre 30].

\bibitem{owl} W3C. OWL: Use Cases and Requirements [Documento en línea]. Disponible: http://www.w3.org/TR/2004/REC-webont-req-20040210/\#onto-def [Consulta: 2010, Diciembre 01].

\bibitem{sparql} W3C. SPARQL Query Language for RDF [Documento en línea]. Disponible: http://www.w3.org/TR/rdf-sparql-query/ [Consulta: 2010, Diciembre 02].

\bibitem{se} Diaz, Selim. Sistemas Expertos. Un paso en la simulación del razonamiento humano [Documento en línea]. Disponible: http://www.monografias.com/trabajos23/sistemas-expertos/sistemas-expertos.shtml [Consulta: 2010, Noviembre 30]

\bibitem{manifiesto} Beck, Kent et al. Manifesto for Agile Software Development [Documento en línea]. Disponible: http://agilemanifesto.org/ [Consulta: 2010, Diciembre 02].

\bibitem{pressman} Pressman, Roger. (2010). Software Engineering: a Practitioner's Approach. New York: McGraw Hill.

\bibitem{sommervile} Sommervile, Ian. (2007). Software Engineering. New York: Pearson Education.

\bibitem{foaf} The Friend of a Friend Project. FOAF Project [Documento en línea]. Disponible: http://www.foaf-project.org [Consulta: 2011, Enero 15].

\bibitem{python} Python Software Foundation. About Python [Documento en línea]. Disponible en: http://www.python.org/about [Consulta: 2011, Agosto 08].

\bibitem{java-joyanes} Joyanes, Luis y Zahonero Ignacio. Programación en Java 2 (2002). Madrid: McGraw Hill.

\bibitem{jena} Jena Community. About Jena [Documento en línea]. Disponible: http://jena.sourceforge.net/index.html [Consulta: 2011, Agosto 08].

\bibitem{sesame} Autor desconocido. About Sesame [Documento en línea]. Disponible: http://www.openrdf.org/about.jsp [Consulta: 2011, Agosto 08].

\bibitem{redland} RedLand Community. RedLand FAQ [Documento en línea]. Disponible: http://librdf.org/FAQS.html [Consulta: 2011, Agosto 08].

\bibitem{cubicweb} CubicWeb Community. The Semantic Web is a Construction Game [Documento en línea]. Disponible http://www.cubicweb.org/ [Consulta: 2011, Agosto 08].

\bibitem{virtuoso} Virtuoso Community. Virtuoso FAQ [Documento en línea]. Disponoble: http://virtuoso.openlinksw.com/dataspace/dav/wiki/Main/VOSVirtuoso6FAQ [Consulta: 2011, Agosto 08].

\bibitem{debian} Debian Community. About Debian [Documento en línea]. Disponible: http://debian.org [Consulta: 2011, Diciembre 20]

\bibitem{techradar} http://www.techradar.com/news/software/operating-systems/10-best-linux-distros-for-2011-704584?artc\_pg=2

\bibitem{norvig-russel} Norvig, Peter y Russell, Stuart. Inteligencia Artificial Un Enfoque Moderno, Segunda Edición (2004). Madrid: Prentice Hall.

\bibitem{aho} Aho, Alfred et al. Compilers Principles, Techniques & Tools, Second Edition (2007). Boston: Pearson Education.

\end{thebibliography}
