\begin{thebibliography}{20}

    \bibitem{swp} Antoniou, G. y Van Harmelen, F. (2003). \textbf{A Semantic Web Primer}. Massachusetts: MIT Press.

    \bibitem{planeta} Cobo, C. y Pardo, H. (2007). \textbf{Planeta 2.0: Inteligencia Colectiva o Medios Fast Food}. México DF: Grup de Recerca d'Interaccions Digitals.

    \bibitem{exito} Éxito Explorador (Agosto 31, 2010). \textbf{Estadísticas Mundiales de Internet} [Datos en línea]. Disponible: http://www.exitoexportador.com/stats.htm [Consulta: 2010, Noviembre 28].

    \bibitem{gbws} W3C. \textbf{Guía Breve de la Web Semántica} [Documento en línea]. Disponible: http://www.w3c.es/divulgacion/guiasbreves/websemantica [Consulta: 2010, Noviembre 28].

    \bibitem{ricardo} Casanova, R. (2010). \textbf{El Modelo Web 2.0}. Presentación sobre el modelo de negocios orientado a la Web 2.0.

    \bibitem{rdfp} W3C. \textbf{RDF Primmer} [Documento en línea]. Disponible: http://www.w3.org/TR/2004/REC-rdf-primer-20040210/ [Consulta: 2010, Noviembre 28].

    \bibitem{rdf} W3C. \textbf{Resource Description Framework (RDF)} [Documento en línea]. Disponible: http://www.w3.org/RDF/ [Consulta: 2010, Noviembre 29].

    \bibitem{rdffaq} GSI. \textbf{RDF Schema (RDFS)} [Presentación en línea]. Disponible: www.gsi.dit.upm.es/~gfer/ssii/RDFS.pdf [Consulta: 2010, Noviembre 30].

    \bibitem{owlfaq} W3C. \textbf{Preguntas Frecuentes del OWL} [Documento en línea]. Disponible: http://www.w3c.es/Traducciones/es/SW/2005/owlfaq [Consulta: 2010, Noviembre 30].

    \bibitem{owl} W3C. \textbf{OWL: Use Cases and Requirements} [Documento en línea]. Disponible: http://www.w3.org/TR/2004/REC-webont-req-20040210/\#onto-def [Consulta: 2010, Diciembre 01].

    \bibitem{sparql} W3C. \textbf{SPARQL Query Language for RDF} [Documento en línea]. Disponible: http://www.w3.org/TR/rdf-sparql-query/ [Consulta: 2010, Diciembre 02].

    \bibitem{se} Diaz, S. (s/f). \textbf{Sistemas Expertos. Un paso en la simulación del razonamiento humano} [Documento en línea]. Disponible: http://www.monografias.com/trabajos23/sistemas-expertos/sistemas-expertos.shtml [Consulta: 2010, Noviembre 30]

    \bibitem{manifiesto} Beck, K. et al. (2001). \textbf{Manifesto for Agile Software Development} [Documento en línea]. Disponible: http://agilemanifesto.org/ [Consulta: 2010, Diciembre 02].

    \bibitem{pressman} Pressman, R. (2010). \textbf{Software Engineering: a Practitioner's Approach}. New York: McGraw Hill.

    \bibitem{sommervile} Sommervile, I. (2007). \textbf{Software Engineering}. New York: Pearson Education.

    \bibitem{foaf} The Friend of a Friend Project. \textbf{FOAF Project} [Documento en línea]. Disponible: http://www.foaf-project.org [Consulta: 2011, Enero 15].

    \bibitem{python} Python Software Foundation. \textbf{About Python} [Documento en línea]. Disponible en: http://www.python.org/about [Consulta: 2011, Agosto 08].

    \bibitem{java-joyanes} Joyanes, L. y Zahonero I. (2002). \textbf{Programación en Java 2}. Madrid: McGraw Hill.

    \bibitem{jena} Jena Community (s/f). \textbf{About Jena} [Documento en línea]. Disponible: http://jena.sourceforge.net/index.html [Consulta: 2011, Agosto 08].

    \bibitem{sesame} Autor desconocido (s/f). \textbf{About Sesame} [Documento en línea]. Disponible: http://www.openrdf.org/about.jsp [Consulta: 2011, Agosto 08].

    \bibitem{redland} RedLand Community (s/f). \textbf{RedLand FAQ} [Documento en línea]. Disponible: http://librdf.org/FAQS.html [Consulta: 2011, Agosto 08].

    \bibitem{cubicweb} CubicWeb Community (s/f). \textbf{The Semantic Web is a Construction Game} [Documento en línea]. Disponible http://www.cubicweb.org/ [Consulta: 2011, Agosto 08].

    \bibitem{virtuoso} Virtuoso Community (s/f). \textbf{Virtuoso FAQ} [Documento en línea]. Disponoble: http://virtuoso.openlinksw.com/dataspace/dav/wiki/Main/VOSVirtuoso6FAQ [Consulta: 2011, Agosto 08].

    \bibitem{debian} Debian Community (s/f). \textbf{About Debian} [Documento en línea]. Disponible: http://debian.org [Consulta: 2011, Diciembre 20]

    \bibitem{techradar} Autor Desconocido (s/f). \textbf{10 Best Linux Distros for 2011} [Documento en línea]. Disponible: http://www.techradar.com/news/software/operating-systems/10-best-linux-distros-for-2011-704584?artc\_pg=2 [Consulta: 2012, Enero 10]

    \bibitem{norvig-russel} Norvig, P. y Russell, S. (2004). \textbf{Inteligencia Artificial Un Enfoque Moderno} (2a. ed). Madrid: Prentice Hall.

    \bibitem{aho} Aho, A. et al. (2007). \textbf{Compilers Principles, Techniques & Tools} (2a. ed). Boston: Pearson Education.

    \bibitem{lucia} Cardoso, L. (2006). \textbf{Sistemas de Base de Datos II, Teoría aplicada para profesores y estudiantes}. Caracas: Publicaciones UCAB.

    \bibitem{papelLuque} Luque, M. et al. (s/f). \textbf{Caracterización Automática de Perfiles de Usuarios basados en Consultas Lingüísticas Multigranulares usando un Algoritmo Genético Multiobjetivo}. España.

\bibitem{ontoclean} Guarino, N. y Welty, C. (s/f). \textbf{An Overview of OntoClean}. Italia.

\bibitem{kuster} Kúster, W y Rauseo, O. (2009). \textbf{Implementar un servicio web que permita crear mashups públicos o privados para ser usados a través de la mensajería de texto celular}. Trabajo Especial de Grado para optar al título de Ingeniero en Informática, Universidad Católica Andrés Bello. Caracas.

\bibitem{ieeeacm} IEEE-CS y ACM (2001). \textbf{Computing Curricula 2001 Computer Science} [Documento en línea]. Disponible: http://www.acm.org/education/education/education/curric\_vols/cc2001.pdf [Consulta: 2011, Marzo 15].

\bibitem{acm} IEEE-CS y ACM (2004). \textbf{Computer Engineering 2004: Curriculum Guidelines for Undergraduate Degree Programs in Computer Engineering} [Documento en línea]. Disponible: http://www.acm.org/education/education/curric\_vols/CE-Final-Report.pdf [Consulta: 2011, Marzo 15].

\bibitem{ieee} IEEE-CS y ACM (2008). \textbf{Computer Science 2008: an Interim Revision of CS 2001} [Documento en línea]. Disponible: http://www.acm.org//education/curricula/ComputerScience2008.pdf [Consulta: 2011 Marzo, 15]

\end{thebibliography}
