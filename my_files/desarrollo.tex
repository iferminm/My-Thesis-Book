\chapter{Desarrollo}
\label{chap:desarrollo}

El desarrollo de este trabajo, se describe siguiendo la metodología planteada anteriormente. Las iteraciones citadas en el \textit{Marco Metodológico} serán descritas a continuación de manera consecutiva:

\section{[Iteración 1]: Investigación y análisis}
Durante esta etapa no se desarrolló ningún tipo de software, es por ello que no se verán historias de usuario, diseño, desarrollo ni pruebas unitarias.

\subsection{Desarrollo de las tareas}
Las tareas de esta iteración se basaron principalmente en la investigación y profundización de los temas relacionados a este trabajo, así como la búsqueda de herramientas para el ensamblaje de la plataforma de desarrollo.

\subsubsection{Investigación de los temas relacionados}
La mayor parte de la investigación realizada fue descrita en el \textit{Marco Teórico} del presente trabajo. El tópico central del presente trabajo resulta ser \textit{La Web Semántica}, por ello, es necesario tomar en cuenta todos los conceptos que se desprenden de dicho tópico. Los conceptos más importantes que se desprenden de \textit{La Web Semántica} son citados a continuación:

\begin{itemize}
\item Ontologías y modelos de conocimiento.
\item Motores de inferencia.
\item Servicios Web.
\end{itemize}

Las \textit{Ontologías} y \textit{Modelos de conocimiento}, son el marco central de la \textit{Web Semántica} 

\newpage
