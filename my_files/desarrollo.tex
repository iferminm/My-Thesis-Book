\chapter{Desarrollo}
\label{chap:desarrollo}

El desarrollo de este trabajo, se describe siguiendo la metodología planteada anteriormente. Las iteraciones citadas en el \textit{Marco Metodológico} serán descritas a continuación de manera consecutiva:

\section{[Iteración 1]: Investigación y análisis}
Durante esta etapa no se desarrolló ningún tipo de software, es por ello que no se verán historias de usuario, diseño, desarrollo ni pruebas unitarias.

\subsection{Desarrollo de las tareas}
Las tareas de esta iteración se basaron principalmente en la investigación y profundización de los temas relacionados a este trabajo, así como la búsqueda de herramientas para el ensamblaje de la plataforma de desarrollo.

\subsubsection{Investigación de los temas relacionados}
La mayor parte de la investigación realizada fue descrita en el \textit{Marco Teórico} del presente trabajo. El tópico central del presente trabajo resulta ser \textit{La Web Semántica}, por ello, es necesario tomar en cuenta todos los conceptos que se desprenden de dicho tópico. Los conceptos más importantes que se desprenden de \textit{La Web Semántica} son citados a continuación:

\begin{itemize}
\item Ontologías y modelos de conocimiento.
\item Motores de inferencia.
\item Servicios Web.
\end{itemize}

Las \textit{Ontologías} y \textit{Modelos de conocimiento}, son el marco central de la \textit{Web Semántica}. Es lo que establece toda la estructura de metadatos en la que se basa el etiquetado e indexado de los recursos que forman parte de la base de conocimientos. Además, establece las relaciones entre las metaentidades que conforman la \textit{Ontología}. Define el vocabulario que modela el dominio del problema a ser resuelto.

Los \textit{Motores de inferencia} son herramientas que, dadas ciertas reglas sobre un \textit{modelo de conocimiento}, son capaces de deducir nueva información y nuevas relaciones entre las metaentidades y, por lo tanto, nuevas relaciones entre los recursos dentro de la base de conocimientos.

Los \textit{Servicios Web}, si bien son un concepto que ya tiene tiempo, cobran especial importancia en la \textit{Web Semántica} pues, cuando se haga efectiva la evolución de la Web 2.0 a la Web 3.0, debe garantizarse total interoperabilidad e independencia de plataformas para que los agentes puedan ser capaces de consultar e intercambiar información y \textit{modelos de conocimiento} de distintas fuentes. Los \textit{servicios} web exponen toda la funcionalidad de un sistema a través de métodos invocables utilizando HTTP sobre TCP. Es por ello que, a través de \textit{Servicios Web}, es posible desarrollar servicios sobre cualquier plataforma y, sin importar en cual otra esté desarrollado, cualquier cliente será capaz de consumir los servicios pues la invocación se encapsula dentro de un protocolo común entre ambas plataformas.

\subsubsection{El problema y los requerimientos}
Del problema general del proyecto, puede plantearse de la siguiente manera: \textit{¿Cómo facilitar el acceso a la información acerca de Ciencias de la Computación a personas interesadas en el área}. De este problema, puede identificarse los siguientes requerimientos:

\begin{itemize}
\item Una interfaz web donde los usuarios puedan interactuar con el sistema.
\item Una interfaz de edición que permita a usuarios autorizados agregar nuevos recursos y extender la base de conocimiento.
\item Una interfaz de edición que permita a usuarios autorizados agregar metainformación nueva y extender el modelo de conocimiento.
\item Un componente de traducción que convierta consultas en lenguaje natural a SPARQL, el lenguaje de consultas sobre RDF/RDFS/OWL.
\item Desarrollo de un modelo de conocimiento del área de Ciencias de la Computación e Ingeniería Informática.
\item Un componente capaz de realizar consultas e inferencias sobre el modelo de conocimiento realizado.
\end{itemize}

\newpage
