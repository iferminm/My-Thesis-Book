\chapter{Desarrollo}
\label{chap:desarrollo}

En el presente cap'itulo se explica en detalle los pasos seguidos para el desarrollo de este T.E.G. Como se describe en el cap\'itulo anterior (Cap'itulo  [\ref{chap:marcometodologico}]) y como se puede apreciar en la figura [\ref{fig:met}] el desarrollo se dividi'o en tres (3) procesos:

\begin{enumerate}
\item Configuraci\'on.
\item Servicios y Gesti\'on de Pesos.
\item Realidad Aumentada.
\end{enumerate}

\section{Proceso de Configuraci'on}

...

\section{Proceso de Servicios y Gesti\'on de Pesos}

...

\renewcommand{\baselinestretch}{1}

\begin{algorithm}[H]
{\footnotesize
\begin{verbatim}
/**
* Obtiene la distancia geodesica entre 2 puntos
* medida en kilometros.
* 
* @author Ricardo Recaredo
*
* @param double $lat1 Latitud inicial
* @param double $long1 Longitud inicial
* @param double $lat2 Latitud final
* @param double $long2 Longitud final
*
* @return double 
*/

function distanciaGeodesica($lat1, $long1, $lat2, $long2){\ldots}
\end{verbatim}
}
\caption{\footnotesize Ejemplo de Comentarios}
\end{algorithm}

\renewcommand{\baselinestretch}{1.5}



\section{Proceso de Realidad Aumentada}
\label{sec:pdra}

\subsection{Especificaci\'on de Funcionalidades}

\newpage

