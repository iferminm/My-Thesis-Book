\chapter{M\'etodo de Ponderaci\'on Lineal (scoring)}
\label{AppendixMPL}

\section{Matriz de ponderaci\'on del Score para el m\'etodo de ponderaci\'on lineal}
Para tener una visi\'on mas amplia del escenario de evaluaci\'on y de la comparativa de los Scores sobre los diferentes criterios para cada una de las alternativas, es recomendable elaborar una Matriz de ponderaci\'on de alternativas basada en el peso asignado a los criterios. 

Lo usual es que en la Matriz De Ponderaci\'on en su forma general[\ref{tbl:matrizPonderacion}], en la primera columna se presenten las alternativas a ser evaluadas y en las siguientes columnas los criterios, dejando la primera fila para identificar los respectivos criterios y los rangos de sus pesos y las restantes casillas de la matriz para realizar la valoraci\'on propiamente dicha, y se conserva la \'ultima columna para completar la evaluaci\'on de cada alternativa, sumando los puntos acumulados por la misma, en su respectiva fila.

\renewcommand{\baselinestretch}{1}

\nuevatabla{matrizPonderacion}
{\footnotesize Matriz De Ponderaci\'on en su forma m\'as general}
{\tiny Fuente: Matrices de Ponderaci\'on para la Evaluaci\'on de Proveedores. 2007}
{p{2cm}p{1.5cm}p{1.5cm}p{0.5cm}p{2cm}p{1.5cm}p{1cm}}
{
\multicolumn{1}{>{\columncolor{vivinicered}}c}{} &
\multicolumn{1}{>{\columncolor{vivinicered}}c}{\footnotesize \textcolor{white}{\textbf{Criterio 1}}} &
\multicolumn{1}{>{\columncolor{vivinicered}}c}{\footnotesize \textcolor{white}{\textbf{Criterio 2}}}& 
\multicolumn{1}{>{\columncolor{vivinicered}}c}{}  & 
\multicolumn{1}{>{\columncolor{vivinicered}}c}{\footnotesize \textcolor{white}{\textbf{Criterio n-1}}} & 
\multicolumn{1}{>{\columncolor{vivinicered}}c}{\footnotesize \textcolor{white}{\textbf{Criterio n }}} & 
\multicolumn{1}{>{\columncolor{vivinicered}}c}{}\\
 
\multicolumn{1}{>{\columncolor{vivinicered}}c}{\footnotesize \textcolor{white}{\textbf{Alternativas}}} & 
\multicolumn{1}{>{\columncolor{vivinicered}}c}{\footnotesize \textcolor{white}{\textbf{Peso x a X}}} & 
\multicolumn{1}{>{\columncolor{vivinicered}}c}{\footnotesize \textcolor{white}{\textbf{Peso y a Y}}} & 
\multicolumn{1}{>{\columncolor{vivinicered}}c}{\footnotesize \textcolor{white}{\textbf{\ldots{}}}} & 
\multicolumn{1}{>{\columncolor{vivinicered}}c}{\footnotesize \textcolor{white}{\textbf{Peso w a W}}}  & 
\multicolumn{1}{>{\columncolor{vivinicered}}c}{\footnotesize \textcolor{white}{\textbf{Peso z a Z}}} & 
\multicolumn{1}{>{\columncolor{vivinicered}}c}{\footnotesize \textcolor{white}{\textbf{Total}}} \\ \hline

\multicolumn{1}{c}{\footnotesize Alternativa 1} &
\multicolumn{1}{c}{\footnotesize P1,1} & 
\multicolumn{1}{c}{\footnotesize P1,2} & 
\multicolumn{1}{c}{\footnotesize \ldots{}} & 
\multicolumn{1}{c}{\footnotesize P1,n-1}  & 
\multicolumn{1}{c}{\footnotesize P1,n} & 
\multicolumn{1}{c}{\footnotesize Total 1} \\ 

\multicolumn{1}{>{\columncolor{cap}}c}{\footnotesize Alternativa 2} & 
\multicolumn{1}{>{\columncolor{cap}}c}{\footnotesize P2,1} & 
\multicolumn{1}{>{\columncolor{cap}}c}{\footnotesize P2,2} & 
\multicolumn{1}{>{\columncolor{cap}}c}{\footnotesize \ldots{}} & 
\multicolumn{1}{>{\columncolor{cap}}c}{\footnotesize p2,n-1}  & 
\multicolumn{1}{>{\columncolor{cap}}c}{\footnotesize P2,n} & 
\multicolumn{1}{>{\columncolor{cap}}c}{\footnotesize Total 2}\\

\multicolumn{1}{c}{\footnotesize \ldots{}} &
\multicolumn{1}{c}{\footnotesize \ldots{}} & 
\multicolumn{1}{c}{\footnotesize \ldots{}} & 
\multicolumn{1}{c}{\footnotesize \ldots{}} & 
\multicolumn{1}{c}{\footnotesize \ldots{}} & 
\multicolumn{1}{c}{\footnotesize \ldots{}} & 
\multicolumn{1}{c}{\footnotesize \ldots{}}\\

\multicolumn{1}{>{\columncolor{cap}}c}{\footnotesize Alternativa m-1} &  
\multicolumn{1}{>{\columncolor{cap}}c}{\footnotesize Pm-1,1} & 
\multicolumn{1}{>{\columncolor{cap}}c}{\footnotesize Pm-1,2} & 
\multicolumn{1}{>{\columncolor{cap}}c}{\footnotesize \ldots{}} & 
\multicolumn{1}{>{\columncolor{cap}}c}{\footnotesize Pm-1,n-1} & 
\multicolumn{1}{>{\columncolor{cap}}c}{\footnotesize Pm-1,n} & 
\multicolumn{1}{>{\columncolor{cap}}c}{\footnotesize Total m-1}\\

\multicolumn{1}{c}{\footnotesize Alternativa m} & 
\multicolumn{1}{c}{\footnotesize Pm,1} & 
\multicolumn{1}{c}{\footnotesize Pm,2} & 
\multicolumn{1}{c}{\footnotesize \ldots{}} & 
\multicolumn{1}{c}{\footnotesize Pm,n-1} & 
\multicolumn{1}{c}{\footnotesize Pm,n} & 
\multicolumn{1}{c}{\footnotesize Total m}\\ \hline
}

\renewcommand{\baselinestretch}{1.5}

En la forma m\'as usual los rangos son todos iguales, es decir x = y = \ldots{} = w = z, en general igual a cero (0) o uno (1) como valor inicial y X = Y = \ldots{} = W = Z = a un valor fijo, generalmente diez (10), veinte (20), cincuenta (50), cien (100) o mil (1000), de acuerdo a lo dif\'icil que puede ser diferenciar entre una alternativa y otra. La otra variante, comnmente usada da valores diferentes a los rangos de cada criterio, en este caso los valores de x, y, w, z, no necesariamente son ceros o unos, y los valores de X, Y, W, Z, suelen ser diferentes, entendi\'endose que entre mayor sea este \'ultimo valor, mayor peso se le desea dar a ese criterio y entre mayor sea la diferencia del mayor valor de un criterio menos el menor valor del mismo (X \textendash{} x, \ldots{}, Z \textendash{} z), se desea ser m\'as diferenciador en cuanto a este criterio. 

