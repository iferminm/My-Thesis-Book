\chapter{Marco Te'orico}
\label{chap:marcoTeorico}

\section{Telefon\'ia M\'ovil}
La telefon\'ia m\'ovil, tambi\'en llamada telefon\'ia celular, b\'asicamente est\'a formada por dos grandes partes: una red de comunicaciones (o red de telefon\'ia m\'ovil) y los terminales (o tel\'efonos m\'oviles) que permiten el acceso a dicha red.\\

El tel\'efono m\'ovil o celular es un dispositivo inal\'ambrico electr\'onico que permite tener acceso a la red de telefon\'ia celular o m\'ovil. Se denomina celular debido a las antenas repetidoras que conforman la red, cada una de las cuales es una c\'elula, si bien existen redes telef\'onicas m\'oviles satelitales.\\

\subsection{GSM}
El sistema GSM, Global System for Mobile communications, es el sistema de telefon\'ia m\'ovil de segunda generaci\'on m\'as extendido por todo el mundo. Se trata a su vez del sistema de telefon\'ia m\'ovil de segunda generaci\'on europeo; la estandarizaci\'on del mismo fue llevada a cabo por la ETSI (European Telecomunications Standard Institute) entre 1982 y 1992.\cite{OriolSallentRoig2003}

El objetivo de este proyecto era poner fin a la incompatibilidad de sistemas en el \'area de las comunicaciones m\'oviles y crear una estructura de sistemas de comunicaciones a nivel europeo que operara en la banda de 900Mhz. \cite{(IEC)}

GSM se dise\~n\'o para incluir una amplia variedad de servicios que incluyen transmisiones de voz y servicios de manejo de mensajes entre unidades m\'oviles o cualquier otra unidad port\'atil.\cite{ClavijoEnricForner;TorrentCuevas}

Para mayor informaci'on, v'ease: Ap'endice \ref{AppendixG}

\section{Toma de Decisiones}
La toma de decisiones es el proceso de convertir informaci\'on en acci\'on.  Es un proceso de identificaci\'on y formulaci\'on de soluciones factibles, evaluaci\'on de las soluciones y selecci\'on de la mejor soluci\'on.  \cite{MauricioEscudey(eds.)1997}

Seg\'un Herbert Simon (1960) es un proceso en el cual se realiza una selecci\'on entre cursos alternativos de acci\'on, basado en un conjunto de criterios, para alcanzar uno o m\'as objetivos. \cite{Herbert1960}

\subsection{Evaluaci\'on Multicriterio}
El concepto gen\'erico de evaluaci\'on multicriterio como conjunto de operaciones espaciales para lograr un objetivo teniendo en consideraci\'on simult\'aneamente todas las variables que intervienen \cite{Barredo1996}, bien sean factores o restricciones \cite{S.Mass1995} sirve de soporte para diversidad de objetivos, frecuentemente relacionados con la toma de decisiones y en ocasiones derivados hacia la evaluaci\'on multiobjetivo cuando entran en juego fuerzas de competencia entre diferentes usos. \cite{Moreno1991}

Los criterios se dice que pueden encontrarse estrictamente en conflicto, lo que se traduce en que el incremento en la satisfacci\'on de uno, implica el decremento en la satisfacci\'on del otro.\cite{Hurtado2005}

\subsection{M\'etodos  de Evaluaci\'on Multicriterio}
Los m\'etodos de evaluaci\'on y decisi\'on multicriterio comprenden la selecci\'on entre un conjunto de alternativas factibles, la optimizaci\'on con varias funciones objetivo simult\'aneas y un \'unico agente decisor, y procedimientos de evaluaci\'on racionales y consistentes.

En general, no existe una alternativa (soluci\'on) que satisfaga y sea preferible en cada una de las funciones objetivo (criterios). Normalmente, se presenta el caso de alternativas factibles, o sea aquellas que cumplen las restricciones, que son mejores que otras en relaci\'on a algunos criterios y que son peores que otras respecto a los restantes criterios.\cite{MauricioEscudey(eds.)1997}

Aquellos problemas en los que las alternativas de decisi\'on son finitas se denominan problemas de decisi\'on multicriterio discretos, mientras que aquellos que poseen un n\'umero infinito de alternativas son considerados problemas de decisi\'on multicriterio continuos.\cite{Hurtado2005}

\subsection{M\'etodos de Evaluaci\'on Multicriterio Discretos}
Los m\'etodos de Evaluaci\'on Multicriterio Discretos se utilizan para realizar una evaluaci\'on y decisi\'on respecto a problemas que, por naturaleza o dise\~no, admiten un n\'umero finito de alternativas de soluci\'on, a trav\'es de:

\renewcommand{\baselinestretch}{1}
\nuevatabla
{psgp}
{\footnotesize M\'etodos de Evaluaci\'on Multicriterio Discretos}
{Fuente: El Proceso de an\'alisis jer\'arquico (AHP) como herramienta para la toma de decisiones en la selecci\'on de proveedores \cite{Hurtado2005}}
{p{3cm}p{10cm}}
{
\multicolumn{1}{>{\columncolor{vivinicered}}p{3cm}}{\textcolor{white}{\textbf{\scriptsize Nombre}}} & 
\multicolumn{1}{>{\columncolor{vivinicered}}p{10cm}}{\textcolor{white}{\textbf{\scriptsize Descripci\'on}}} \\ \hline

\multicolumn{1}{p{3cm}}{\scriptsize Alternativas} & 
\multicolumn{1}{p{10cm}}{\scriptsize Conjunto finito de soluciones factibles que cumplen con las restricciones posibles o previsibles.}\\

\multicolumn{1}{>{\columncolor{cap}}p{3cm}}{\scriptsize Criterios o Atributos} & 
\multicolumn{1}{>{\columncolor{cap}}p{10cm}}{\scriptsize Caracter\'isticas que se utilizan para describir y/o evaluar cada una de las alternativas disponibles. Pueden ser cuantitativas o cualitativas.}\\

\multicolumn{1}{p{3cm}}{\scriptsize Objetivos} & 
\multicolumn{1}{p{10cm}}{\scriptsize Los objetivos son aspiraciones que indican direcciones de perfeccionamiento de los atributos seleccionados. Est\'an asociados con los deseos y preferencias del decisor.}\\

\multicolumn{1}{>{\columncolor{cap}}p{3cm}}{\scriptsize Matriz de ponderaci\'on} & 
\multicolumn{1}{>{\columncolor{cap}}p{10cm}}{\scriptsize Resume la evaluaci\'on de cada alternativa conforme a cada criterio; una valoraci\'on (precisa o subjetiva) de cada una de las soluciones a la luz de cada uno de los criterios, la escala de medida de las evaluaciones puede ser cuantitativa o cualitativa, y las medidas pueden expresarse en escalas cardinal (raz\'on e intervalo), ordinal, nominal, y probabil\'istica.}\\

\multicolumn{1}{p{3cm}}{\scriptsize Metodolog\'ia o modelo de agregaci\'on de preferencias} & 
\multicolumn{1}{p{10cm}}{\scriptsize En una s\'intesis global; ordenaci\'on, clasificaci\'on, partici\'on, o jerarquizaci\'on de dichos juicios para determinar la soluci\'on que globalmente recibe las mejores evaluaciones.}\\
}
\renewcommand{\baselinestretch}{1.5}

Los principales m\'etodos de evaluaci\'on y decisi\'on multicriterio discreto son: Ponderaci\'on lineal, Utilidad multiatributo, Relaciones de superaci\'on y el an\'alisis del proceso jer\'arquico.\cite{Hurtado2005}

\subsection{M\'etodo de Ponderaci\'on Lineal (scoring)}
Es un m\'etodo con una fundamentaci\'on te\'orica ortodoxa y directa, que permite abordar situaciones de incertidumbre o con modestos niveles de informaci\'on, y que consiste en construir una funci\'on de valor para cada alternativa. El m\'etodo de Ponderaci\'on Lineal supone la transitividad de preferencias o la comparabilidad. Es un m\'etodo completamente compensatorio, y puede resultar dependiente, y manipulable, de la asignaci\'on de pesos a los criterios o de la escala de medida de las evaluaciones. Es un m\'etodo ampliamente difundido.
Las etapas del m\'etodo son las siguientes:
\begin{enumerate}
\item Identificar el objetivo general del problema.
\item Identificar las alternativas.
\item Listar los criterios a emplear en la evaluaci\'on.
\item Asignar una ponderaci\'on para cada uno de los criterios.
\item Calcular el Score para cada una de las alternativas.
\item Ordenar las alternativas en funci\'on del Score. La alternativa con el Score m\'as alto representa la alternativa a recomendar.
\end{enumerate}

Modelo para calcular el Score:

\begin{eqnarray}
	S_j=\sum_{j}w_ir_i{}_j  
\end{eqnarray} 

Donde:
\begin{itemize}
\item la $r_i{}_j$: Valor de la alternativa j en funci\'on del criterio i.
\item la $w_i$: ponderaci\'on para cada criterio i.
\item la $S_j$: Score para la alternativa j.
\end{itemize}

Para mayor informaci'on, v'ease: Ap'endice \ref{AppendixMPL}

\subsection{Indicador}
La palabra indicador seg\'un la definici\'on del Diccionario de la Real Academia, se utiliza para ``significar algo con indicios y se\~nales''.\cite{Espanola}

Mientras que la OMS\footnote{\textbf{OMS}: Del Espa\~nol:  Organizaci\'on Mundial de la Salud.}  define a los indicadores como variables que intentan medir u objetivar en forma cuantitativa o cualitativa, sucesos para as\'i, poder respaldar acciones y decisiones, evaluar logros y metas. Son una medida sustituta de informaci\'on que permite calificar un concepto abstracto y para medir o comparar los resultados efectivamente obtenidos en la ejecuci\'on de un proyecto, programa o actividad.\cite{OMS1981}

\newpage