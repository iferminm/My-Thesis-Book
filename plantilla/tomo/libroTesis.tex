\documentclass[letterpaper,12pt,oneside,openany, spanish, oldfontcommands]{memoir}

\usepackage[spanish,activeacute,es-noshorthands]{babel}
\usepackage[latin1]{inputenc}
\usepackage{setspace}
\usepackage{tabularx}
\usepackage{anysize}
\usepackage{algorithm}
\usepackage{algorithmic}
%archivo para que los algoritmos me salgan en español
\input{spanishAlgorithmic}
\usepackage{array}
 %Inclusi\'on de gr\'aficos al lado de texto
\usepackage{wrapfig}
\usepackage{colortbl}
\usepackage{graphicx}
\usepackage[left=4cm,top=3cm,right=3cm,bottom=3cm]{geometry}
\usepackage{slashbox}
\usepackage{supertabular}
\usepackage{longtable}
\usepackage{rotating}
%para usar multiples filas
\usepackage{multirow}
%para usar multiples columnas
\usepackage{dcolumn}
\usepackage{pdflscape}
\usepackage{amsmath}
%evitar el reset de los footnotes entre cada cap'itulo
\usepackage{remreset}
\makeatletter\@removefromreset{footnote}{chapter}\makeatother
%footnotes en la parte de abajo de la pagina
\usepackage[bottom]{footmisc}
\usepackage{titlesec}
\usepackage{float}
%Incluir bibliograf\'ia en el \'indice
\usepackage{tocbibind} 
\usepackage{pdflscape}
\usepackage{amsmath}
% paquete para el uso de simbolos como el checkmark
\usepackage{bbding}
%Inclusi\'on de gr\'aficos al lado de texto
\usepackage{wrapfig} 
\usepackage{colortbl} 
\usepackage{titlesec}
\usepackage{float}
\usepackage{tabularx}
\usepackage{anysize}
\usepackage{array}
\usepackage{color}
%para los simbolos usados en las vinetas
\usepackage{pifont}
\usepackage[dvips,final]{epsfig}
\definecolor{vivinicered}{RGB}{0,104,166}
\definecolor{cap}{RGB}{204,230,255}
% Para tener marcadores por secciones y subsecciones
\usepackage[pdftex=true,colorlinks=true,plainpages=false]{hyperref}
\hypersetup{
    colorlinks,
    citecolor=vivinicered,
    filecolor=vivinicered,
    linkcolor=vivinicered,
    urlcolor=vivinicered,
}

%\usepackage[table,svgnames,hyperref]{xcolor}
\usepackage{calc}
\usepackage{ifthen}
\usepackage{pgf,tikz}
\usetikzlibrary{shapes,arrows,calc,positioning,shadows}

%%%%%%%%	COMO SE VEN LOS CAPITULOS	%%%%%%%%%%
\usepackage{color,calc,graphicx,soul,fourier}
\makeatletter
\newlength\dlf@normtxtw
\setlength\dlf@normtxtw{\textwidth}
\def\myhelvetfont{\def\sfdefault{mdput}}
\newsavebox{\feline@chapter}
\newcommand\feline@chapter@marker[1][4cm]{%
\sbox\feline@chapter{%
\resizebox{!}{#1}{\fboxsep=1pt%
\colorbox{vivinicered}{\color{white}\bfseries\sffamily\thechapter}%
}}%
\rotatebox{90}{%
\resizebox{%
\heightof{\usebox{\feline@chapter}}+\depthof{\usebox{\feline@chapter}}}%
{!}{\scshape\so\@chapapp}}\quad%
\raisebox{\depthof{\usebox{\feline@chapter}}}{\usebox{\feline@chapter}}%
}
\newcommand\feline@chm[1][4cm]{%
\sbox\feline@chapter{\feline@chapter@marker[#1]}%
\makebox[0pt][l]{% aka \rlap
\makebox[1cm][r]{\usebox\feline@chapter}%
}}
\makechapterstyle{daleif1}{
\renewcommand\chapnamefont{\normalfont\Large\scshape\raggedleft\so}
\renewcommand\chaptitlefont{\normalfont\huge\bfseries\scshape\color{vivinicered}}
\renewcommand\chapternamenum{}
\renewcommand\printchaptername{}
\renewcommand\printchapternum{\null\hfill\feline@chm[2.5cm]\par}
\renewcommand\afterchapternum{\par\vskip\midchapskip}
\renewcommand\printchaptertitle[1]{\chaptitlefont\raggedleft ##1\par}
}
\makeatother
\chapterstyle{daleif1}
%%%%%%%%%%%%%%%%%%%%%%%%%%%%%%%%%%%%%%%%%%%%%%%%%%%%%%%%%%%%%%%%%%%%%%%%%%%%%%%%%%%%%%%%%%%%%%%%

%evitar el reset de los footnotes entre cada cap'itulo
\usepackage{remreset}
\makeatletter\@removefromreset{footnote}{chapter}\makeatother

%footnotes en la parte de abajo de la pagina
\usepackage[bottom]{footmisc}

\setcounter{secnumdepth}{3}
\setcounter{tocdepth}{4}

\setcounter{secnumdepth}{3}
% el nivel de detalle del indice. En 3 salen hasta el subsubsection
\setcounter{tocdepth}{3}

% espaciado de 1.5 entre lineas
\renewcommand{\baselinestretch}{1.5}

\AtBeginDocument{
  \def\labelitemi{\ding{224}}
  \def\labelitemii{\ding{223}}
}

%%%%%%%%	COMANDOS PARA IMAGENES PNG	%%%%%%%%
% \pngfig
% Usado para introducir imagenes adentro del directorio img/ cuya extension sea .png
% Float -> valor es el porcentaje de tama~no, 
% String -> nombre del archivo (funciona con archivos con nombres sin espacios)
% String -> Descripci'on que va a ir debajo de la imagen introducida
% String -> Fuente de la figura
% \pngfig[.5]{archihbhjv{descripcion de la imagen introducida} 
\newcommand{\pngfig}[4][1.0]{
\begin{figure}[H]
\begin{centering}
\scalebox{#1}{\includegraphics{/Users/vivianatrujillo/tesis/plantilla/imagenes/#2.png}}
\par\end{centering}
\begin{centering} 
\caption{\footnotesize #3} 
\label{fig:#2}
\par\end{centering}
\centering{}{\tiny #4}
\end{figure}
}

\newcommand{\pngfigrotada}[4][1.0]{
 \begin{figure}[H]
 \centering
 \vspace{2mm}
 \scalebox{#1}{\begin{sideways} \includegraphics{/Users/vivianatrujillo/tesis/plantilla/imagenes/#2.png} \end{sideways} }
 \caption{\footnotesize #3}
 \label{fig:#2}
{\tiny
 #4}
 \end{figure}
}
%%%%%%%%	COMANDOS PARA IMAGENES JPG	%%%%%%%%
\newcommand{\jpgfig}[4][1.0]{
\begin{figure}[H]
\begin{centering}
\scalebox{#1}{\includegraphics{/Users/vivianatrujillo/tesis/plantilla/imagenes/#2.jpg}}
\par\end{centering}
\begin{centering}
\begin{quote} 
\caption{\footnotesize #3}
\label{fig:#2}
\part\end{centering}
\centering{}{\tiny #4}
\end{quote}
\end{figure}
}

% Igual que las anteriores pero esta vez coloca la imagen rotada 90 grados (landscape)
\newcommand{\jpgfigrotada}[4][1.0]{
 \begin{figure}[H]
 \centering
 \vspace{2mm}
 \scalebox{#1}{\begin{sideways} \includegraphics{/Users/vivianatrujillo/tesis/plantilla/imagenes/#2.jpg} \end{sideways} }
 \caption{\footnotesize #3}
 \label{fig:#2}
{\tiny
 #4}
 \end{figure}
}

%%%%%%%%	COMANDOS PARA TABLAS	%%%%%%%%
% \nuevatabla
% Inserta una nueva tabla centrada y con los atributos.
% String -> Nombre de la tabla para luego ser referenciada (Sin espacios)
% String -> Texto descriptivo de la tabla que aparecer'a debajo de la misma.
% String -> Fuente elaboraci'on de la tabla
% Opciones de las columnas introducidas
% EJEMPLO: \nuevatabla{resumenDecripcionesAtributosCalidad}{Resumen de descripciones de atributos de calidad}{Fuente: Elaboraci'on propia}{|c|p{10cm}|}{\textbf{Atributo de Calidad} & \textbf{Descripci'on} \\ \hline}
% Ojo, recordar que las opciones de posici'on de la tabla se est'an colocando por defecto desde este comando y no son escogidos en la invocaci'on. Recordar qu'e:
%b: Intentara ponerla en el fondo de la pagina
%h: Intentara ponerla en la misma posicion en que se encuentra en el codigo fuente
%t: Intentara ponerla al principio de la pagina
%p: La pondra en una pagina que contenga solo elementos flotantes
%!: Ignorar la mayora de las restricciones impuestas por LATEX
\newcommand{\nuevatabla}[5]{
{\scriptsize 
\begin{table}[H] 
\setlength{\arrayrulewidth}{1pt}
\begin{center}
\begin{tabular}{#4}\hline
#5
\hline
\end{tabular}
\begin{quote} 
\caption{\footnotesize #2}
\label{tbl:#1}
\begin{center}
{\tiny #3}
\end{center}
\end{quote}
\end{center}
\end{table}
}
}

% lo mismo que la anterior pero con la tabla rotada 90 grados
\newcommand{\nuevatablarotada}[5]{
\begin{landscape}
\begin{table}[H] \footnotesize %[!hbt]%[htdp]
\setlength{\arrayrulewidth}{1pt}
\begin{center}
\caption{\footnotesize #2}
{\tiny
#3\\
\begin{tabular}{#4}\hline
#5}
\label{tbl:#1}
\end{tabular}
\end{center}
\end{table}
\end{landscape}
}

% \nuevalongtabla
%Cuando la tabla es muy grande y no cabe en una p'agina es necesario el uso de esta funci'on
% es igual a la \nuevatabla, pero garantiza que se pueda pasar a m'as de una p'agina.
%1 label
%2 caption de la tabla
%3 fuente
%4 la estructura
%5 t'itulo que se debe mantener en otras hojas
%6 contenido
\newcommand{\nuevalongtabla}[6]{
\begin{longtable}[H] 	 {#4}
\hline
\endfirsthead
\hline
#5 \\ 
\hline 
\endhead 
#6
\hline 
\caption{\footnotesize #2 - #3}
\label{tbl:#1}
\end{longtable}
}

% \nuevalongtablarotada
%Cuando la tabla es muy grande y no cabe en una p'agina es necesario el uso de esta funci'on
% es igual a la \nuevatabla, pero garantiza que se pueda pasar a m'as de una p'agina.
%1 label
%2 caption de la tabla
%3 fuente
%4 la estructura
%5 t'itulo que se debe mantener en otras hojas
%6 contenido
\newcommand{\nuevalongtablarotada}[6]{
\begin{landscape}
\begin{longtable}[H]{#4} 
\caption{\footnotesize #2}
#3\\
\hline
\endfirsthead
\hline
#5 \\ 
\hline 
\endhead 
#6
\label{tbl:#1}
\end{longtable}
\end{landscape}
}

%%%%%%%%%%%%%%%%%%%%%%%%%%%%%%%%%%%%%%%%%%%% COMIENZO DEL DOCUMENTO %%%%%%%%%%%%%%%%%%%%%%%%%%%%%%%%%%%%%%%%%%%%%%%%%%%%%%%%%%%%%%

\begin{document}

\renewcommand{\contentsname}{'INDICE DE CONTENIDO}
\renewcommand{\bibname}{Bibliograf\'ia} % Para que ya no diga Bibliograf'ia sino diga Referencias
\renewcommand{\tablename}{Tabla} % Los Captions de tablas salen con la palabra Tabla y no con la palabra Cuadro
\renewcommand{\listtablename}{'INDICE DE TABLAS} % Para que diga 'indice de tablas y no 'indice de cuadros
\renewcommand{\listfigurename}{'INDICE DE FIGURAS}
\renewcommand{\chaptername}{CAP'ITULO}
    
\parskip 6pt

\pagenumbering{roman}
\pagestyle{empty}

\begin{center}
\parbox{4cm}{\includegraphics[width=0.16\textwidth]{/home/israelord/Dissertation/Book/my_files/images/escudoUcab.jpg}}\parbox{9cm}{\begin{center}
UNIVERSIDAD CATÓLICA ANDRÉS BELLO \\
FACULTAD DE INGENIERÍA \\
ESCUELA DE INGENIERÍA INFORMÁTICA \\
\end{center}}

\vspace{\fill}


\begin{LARGE}
Desarrollo de una ontología y un buscador utilizando tecnologías de Web Semántica
\end{LARGE}

\vspace{\fill}

TRABAJO ESPECIAL DE GRADO\\
presentado ante la\\
UNIVERSIDAD CATOLICA ANDRES BELLO\\
como parte de los requisitos para optar al t\'itulo de\\
INGENIERO EN INFORM\'ATICA

\vspace{\fill}

{
\begin{tabular}{cl}
REALIZADO POR & \textbf{Israel Fermín Montilla}\tabularnewline
\multicolumn{2}{c}{}\tabularnewline
PROFESOR GUIA & \textbf{Ph.D: Wilmer Pereira}\tabularnewline
\multicolumn{2}{c}{}\tabularnewline
FECHA & Caracas, septiembre de 2011\tabularnewline
\end{tabular}
}


\end{center}

\chapter*{AGRADECIMIENTOS}
\addcontentsline{toc}{chapter}{AGRADECIMIENTOS} 

En primer lugar, a mi familia que siempre me ha apoyado en mis metas y mis locuras. A mi papá y a mi mamá por estar siempre pendientes de mi formación académica y personal y apoyarme tanto en los buenos como en los malos momentos de la carrera. A Giselle por todo su cariño y su apoyo incondicional para superar las horas de dedicación a este trabajo. A mi tutor, el Prof. Wilmer Pereira, por creer en este proyecto desde el comienzo y a mi amigo el Ing. Carlos Pérez, por su apoyo incondicional durante la realización de este trabajo.

Una mención especial al Prof. Alberto Durán Meza (UJMV), por su apoyo invaluable durante el ciclo básico de la carrera.

A todos los profesores de la Universidad Católica Andrés Bello con los cuales tuve la oportunidad de compartir en las aulas y muchas veces fuera de ellas con un trato menos formal y más amistoso, muy especialmente a: Carlos Barroeta, Lúcia Cardoso, Rodolfo Campos, Darío León y Ricardo Casanova. 

A mis alumnos de preparaduría, especialmente a: Aileen Posadas, Marcos Akerman, Héctor Sam y Juan Perozo.

A mis compañeros y amigos que me acompañaron semestre a semestre en las buenas y en las malas: Gerardo Barcia, Jonathan Trujillo, Khaterine Castellano, Viviana Trujillo, Ronald Oribio y Freddy Florenca.

Finalmente, a todos quienes, de alguna u otra manera me ayudaron,
Muchas Gracias.

\begin{flushright}
\textit{- Israel Fermín Montilla}
\end{flushright}


\pagestyle{plain}

\begin{SingleSpace}
\tableofcontents

\newpage

\listoffigures

\newpage

\listoftables

\newpage

\listofalgorithms
\end{SingleSpace}	

\pagenumbering{arabic}
\pagestyle{empty}

\makepagestyle{ruled}
\makeoddfoot{ruled}{}{}{\thepage}
\makeheadrule{ruled}{\textwidth}{\normalrulethickness}
\makeevenhead{ruled}{\scshape\leftmark}{}{} % small caps
\makeoddhead{ruled}{}{}{\rightmark}
\pagestyle{ruled}


\chapter*{SINOPSIS}
\addcontentsline{toc}{chapter}{RESUMEN} 

Toda la paja en una página.

\chapter*{Introducci\'on}
\label{chap:introduccion}
\addcontentsline{toc}{chapter}{Introducción} 

Internet, la autopista de la información, un enorme grafo con información esperando ser descubierta, conocimiento esperando ser asimilado, servicios esperando ser utilizados, documentos esperando ser leídos, en fin, recursos esperando ser explotados. Ciertamente, para muchos, Internet se ha convertido en un medio primario de entretenimiento, aprendizaje o, incluso, un tópico de investigación. Esta revolución tecnológica ha causado gran impacto en el mundo y ha cambiado, incluso, la manera tradicional de hacer las cosas.

Antes de la aparición de Internet: las empresas debían pagar costosos espacios en radio y TV para poder desplegar su maquinaria publicitaria, dejando a las pequeñas empresas, que no pueden pagar dichos espacios, en desventaja, los políticos debían ir casa por casa hablando con la gente, llevar a cabo apariciones públicas en mítines canales de TV y entrevistas de radio para que su mensaje llegara, los artistas dependías de grandes disqueras para dar a conocer su trabajo y su música y pudiera seguir dando cientos de ejemplos. Todo eso ha cambiado desde la aparición de Internet y, más aún, desde la aparición de lo que Tim O'Reilly llamó la \textit{Web 2.0}, esa red interactiva en la que cualquiera puede generar contenido y la información se encuentra a sólo un click de distancia, ahora es posible realizar campañas publicitarias ``dos punto cero'', desde la comodidad de una habitación u oficina, incluso, dar soporte y apoyo a campañas políticas como la de Barack Obama para la Presidencia de los Estados Unidos en 2008, o la menos exitosa de Diego Arria en Venezuela, compitiendo en 2012 por la candidatura a la Presidencia de la República, esta última se apoyó, principalmente, en las redes sociales como Facebook y Twitter, realizando entrevistas y conferencias a través del servicio de TwitCam para dar a conocer sus propuestas, quedando en cuarto lugar logrando cerca de 135.000 votos.

Ciertamente Internet ha cambiado la manera de hacer investigación, hace 20 años para conseguir información acerca de algún tema, el único recurso disponible era ir a una biblioteca y consultar los libros que allí ofrecen, ahora, para conseguir información acerca del mismo tópico basta con ir a \textit{Wikipedia} para tener una idea general y realizar algunas búsquedas para profundizar un poco más, pero, ¿podemos ver la Internet como una gran base de datos?, una base de datos constituye, en su definición más sencilla, un conjunto de datos relacionados entre sí que tienen un significado para alguien (Cardoso, 2006). Ciertamente, los recursos disponibles en la Internet tienen significado para quien los busca, pero, ¿están relacionados entre sí?, para que estén relacionados, debe haber cierta estructura y orden en los datos, ¿tiene Internet esa estructura y ese orden?, más allá de eso, ¿será posible dar orden y estructura a la Internet para manipularla como una gran base de datos?

La respuesta, afirmativa o negativa, a todas esas interrogantes están en \textit{La Web Semántica} o, como algunos la llaman, \textit{Web 3.0}, que busca dar una estructura flexible a los recursos que hacen vida dentro de la \textit{gran autopista de la información} que constituye la WWW. En la \textit{Web 2.0}, el tema álgido consiste en la manera como se modelan o se presentan los datos (Kúster y Rauseo, 2009), en la \textit{Web 3.0}, el tema será la manera como se describen y se relacionan los datos (recursos) entre sí, haciéndolos relevantes para el usuario en la medida en que su descripción se ajuste a la consulta realizada.

El presente trabajo está estructurado de la siguiente manera:

El Capítulo 1, presenta el planteamiento del problema, los objetivos y la justificación de esta investigación, el Capítulo 2, presenta todo el Marco Teórico que da basamento a este trabajo. Seguidamente, el Capítulo 3, describe la metodología utilizada durante el desarrollo de las actividades necesarias para la realización de este proyecto, el Capítulo 4, describe las actividades realizadas durante el desarrollo. A continuación, el Capítulo 5 presenta los resultados obtenidos como producto de la parte práctica de este trabajo, el Capítulo 6 presenta las Conclusiones y una serie de Recomendaciones para trabajos futuros enmarcados dentro de esta línea de investigación. Finalmente se muestran las Referencias Bibliográficas y los Apéndices.

\end{itemize}

\chapter{Problema}
\label{chap:problema}

%%%%%%%%%%PLANTEAMIENTO DEL PROBLEMA%%%%%%%%%%%%%%

\section{Planteamiento del Problema }

....

Debido a esta potencial problem\'atica y a los retrasos ....

...

%%%%%%%%%%%OBJETIVOS%%%%%%%%%

\section{Objetivo General }

Desarrollar ...

\section{Objetivos Espec\'ificos }

\begin{enumerate}
	\item ...
	\item ...
	\item ...
\end{enumerate}

%%%%%%%%%%%ALCANCE%%%%%%%%%%%

\section{Alcance }

Este Trabajo Especial de Grado da como resultado una aplicaci\'on ...

%%%%%%%%%LIMITACIONES%%%%%%%

\section{Limitaciones }
...
%%%%%%%%%JUSTIFICACION%%%%%%%

\section{Justificaci\'on }

...
\chapter{Marco Te'orico}
\label{chap:marcoTeorico}

Copiar y pegar lo mismo de la propuesta, pero extender
\newpage

\chapter{Marco Metodológico}
\label{chap:marcometodologico}

Las metodologías de desarrollo tradicionales, parecieran haber sido diseñadas para proyectos extensos, con equipos numerosos, en los que se goza de roles bien definidos y que, cada rol, cumple con una labor específica en cada etapa del ciclo de vida.

La realidad de este proyecto es otra, sólo se cuenta con una persona para realizar el trabajo de todos los roles, además, el tiempo de entrega es limitado y resulta, además, conveniente entregarlo lo antes posible. Es por ello que se propone trabajar con una metodología Iterativa Incremental bajo el esquema ágil. Los preceptos del esquema de trabajo enmarcado en el Desarrollo Ágil, fueron definidos por Kent Beck año 2001[13] en un documento denominado El Manifiesto Ágil (The Agile Manifesto), este manifiesto se cita a continuación:

``Estamos descubriendo nuevas maneras de desarrollar software tanto por nuestra propia experiencia como ayudando a terceros. A través de esta experiencia hemos aprendido a valorar:

\begin{itemize}
\item \textbf{Individuos e interacciones} sobre procesos y herramientas.
\item \textbf{Software que funciona} sobre documentación exhaustiva.
\item \textbf{Colaboración con el cliente} sobre negociación de contratos.
\item \textbf{Responder ante el cambio} sobre el seguimiento de un plan.
\end{itemize}

Esto es, aunque los elementos a la derecha tienen valor, nosotros valoramos por encima de ellos a los que están a la izquierda".

El Trabajo Especial de Grado, tiene un carácter investigativo y aplicativo, es decir, debe investigarse acerca del tema y realizar un pequeño aporte al área de investigación mediante el desarrollo de algún producto final. Por ello, dado el carácter de un TEG, todos y cada uno de los principios definidos en el Manifiesto Ágil de Kent Beck tienen sentido en un proyecto de este tipo pues, debe valorarse ``Individuos e interacciones sobre procesos y herramientas", esto es, debe valorarse más la interacción entre el tesista y el tutor que los procesos y herramientas necesarios para ello, por ejemplo. Debe darse más valor al ``Software de funciona sobre la documentación exhaustiva", en este caso, para la presentación final, debe mostrarse algo funcional, si bien la documentación es importante, no se hará de manera exhaustiva, únicamente lo necesario para poner orden en el proyecto y para que, posteriormente, si el presente trabajo resulta de interés para alguien, pueda entender qué fue lo que se hizo. ``Colaboración con el cliente sobre negociación de contratos", si bien esto no tiene mucha pertinencia al hablar de un TEG, puede interpretarse como que la Escuela de Ingeniería Informática y el Tesista deberían colaborar pues ambos persiguen un objetivo común: innovar. Finalmente, "Responder ante el cambio sobre el seguimiento de un plan" pues al tratarse de un proyecto de investigación, no todos los requerimientos están claramente definidos ni pueden predecirse en su totalidad, es por ello que conforme se va investigando y despejando la incertidumbre en algunos temas, pueden ir surgiendo nuevas cosas que no se esperaban y se debe estar preparado para responder y adaptar el plan a las nuevas condiciones.

Existen numerosas metodologías y ciclos de vida basados en el modelo ágil, para este proyecto en particular se propone utilizar una metodología basada en Scrum para organizar el proyecto y XP (Programación Extrema o eXtreme Programming en inglés) para la organización de cada una de las iteraciones.

\section{Descripción de la Metodología Planteada}

Para describir la metodología que se utilizará para el desarrollo del proyecto, es necesario primero estudiar las dos metodologías antes mencionadas:

\subsection{SCRUM}

SCRUM es una metodología creada por Jeff Sutherland y su equipo de desarrollo a principios de la década de 1990 [14]. Los principios de SCRUM están enmarcados dentro del Manifiesto Ágil y es un proceso que lleva el Desarrollo de Software a través de las siguientes actividades: requerimientos, análisis, diseño, evolución y entrega. Cada una de esas actividades son realizadas dentro de un patrón de trabajo llamado Sprint, todo el trabajo realizado dentro de un Sprint es adaptado al problema y frecuentemente modificado por el equipo de desarrollo a medida que las condiciones van cambiando.

SCRUM hace énfasis en la utilización de procesos de software que son efectivos en proyectos con tiempos cortos de entrega y requerimientos cambiantes. Esos procesos, en general, definen dos grandes actividades de desarrollo:

\begin{itemize}
\item \textbf{Backlog:} el backlog, constituye una lista priorizada de requerimientos que agregan valor de negocio al producto, estos requerimientos pueden ser agregados en cualquier momento y, de esta manera, se introducen los cambios en el proyecto [14]. El backlog puede ser, además, modificado en cualquier momento para adaptar las prioridades a los cambios del negocio.
\item \textbf{Sprints:} los Sprints constituyen paquetes de trabajo que son necesarios para desarrollar un requerimiento o un conjunto de ellos [14]. Los cambios no pueden ser introducidos dentro de un Sprint, de esta manera se asegura que el trabajo que se realiza es estable, pues los requerimientos que se seleccionan para ser desarrollados en un Sprint ya deben estar definidos y se debe tener la certeza de que, en caso de cambiar, será manejable.
\item \textbf{Scrum Meetings:} son reuniones cortas que, usualmente, se realizan todos los días durante un sprint [14]. Durante los Scrum Meetings se responden a tres preguntas básicas:
\begin{itemize}
\item ¿Qué has hecho desde la última reunión?.
\item ¿Cuáles obstáculos has encontrado?.
\item ¿Cuál es tu planificación hasta la próxima reunión?.
\end{itemize}
\item \textbf{Demos:} se van entregando los resultados de las funcionalidades desarrolladas al cliente para que puedan ser evaluados [14]. De esta manera, se va entregando Software Listo y Funcional que agrega valor al negocio del cliente en cada iteración.
\end{itemize}

De esta manera, SCRUM permite que los equipos trabajen de manera eficiente y estable en proyectos en los que la incertidumbre siempre está presente.

\subsection{eXtreme Programming}

La metodología eXtreme Programming (XP) es la más utilizada y la más conocida de las metodologías ágiles [15]. Fue inicialmente concebida por Kent Beck a finales de la década de 1980 y se enfoca en hacer énfasis en la etapa de implementación del ciclo de vida, tal como su nombre lo indica, en la parte de programación o codificación, tomando como buena práctica la programación en parejas, de esta manera, mientras uno de los desarrolladores codifica, el otro está observando y advirtiendo de errores a quien escribe el código.

Kent Beck definió cinco valores para establecer las bases de XP como metodología. Cada uno de esos valores se utilizan para poner en marcha las actividades, acciones y tareas de XP [14]:

\begin{itemize}
\item \textbf{Comunicación}: entre los clientes y los desarrolladores. XP valora más la comunicación informal por ser más rápida en comparación con volúmenes enormes de documentación como medio de comunicación principal.
\item \textbf{Simplicidad}: XP restringe a los desarrolladores a realizar el diseño y la implementación de código que satisfaga los requerimientos actuales en vez de pensar en factores futuros, si el diseño puede ser mejorado, puede ser modificado posteriormente.
\item \textbf{Retroalimentación}: durante un proyecto en necesario muchas veces regresar a una etapa previa y modificar algo para que los requerimientos actuales funcionen sin alterar los que ya fueron implementados. El cambio es algo constante en todo proyecto, pero para poder hacerlo correctamente, se necesita retroalimentación.
\item \textbf{Coraje}: implementar código sin miedo a las consecuencias, pero siempre de manera coherente.
\end{itemize}

En eXtreme Programming, no se libera una pieza de software hasta que está totalmente funcional y probada, para asegurar que el software funciona, se desarrollan Pruebas Unitarias. 

La metodología propuesta para el desarrollo del presente proyecto, toma la organización de SCRUM y los valores y la filosofía basada en la codificación de XP para llevar a cabo las tareas y actividades concernientes al desarrollo del producto final de este trabajo, obviando la exigencia de XP acerca de la programación en parejas ya que el presente trabajo es realizado por sólo una persona.

%TODO: Terminar según se vea necesario
\subsection{Plan General de Trabajo}
Se plantea, organizar el desarrollo en las siguientes iteraciones, con una duración de tres a cuatro semanas cada una:

\begin{enumerate}
\item Investigación y análisis.
\item Diseño de la arquitectura del sistema.
\item Establecimiento del ambiente de desarrollo.
\item Anlálisis y desarrollo de la ontología.
\item Integración del modelo de conocimiento con el API seleccionado.
\item Desarrollo del traductor de lenguaje natural a SPARQL.
\item Desarrollo del motor de búsqueda.
\item Desarrollo e integración con las vistas.
\end{enumerate}

\newpage


\chapter{Desarrollo}
\label{chap:desarrollo}

En el presente cap'itulo se explica en detalle los pasos seguidos para el desarrollo de este T.E.G. Como se describe en el cap\'itulo anterior (Cap'itulo  [\ref{chap:marcometodologico}]) y como se puede apreciar en la figura [\ref{fig:met}] el desarrollo se dividi'o en tres (3) procesos:

\begin{enumerate}
\item Configuraci\'on.
\item Servicios y Gesti\'on de Pesos.
\item Realidad Aumentada.
\end{enumerate}

\section{Proceso de Configuraci'on}

...

\section{Proceso de Servicios y Gesti\'on de Pesos}

...

\renewcommand{\baselinestretch}{1}

\begin{algorithm}[H]
{\footnotesize
\begin{verbatim}
/**
* Obtiene la distancia geodesica entre 2 puntos
* medida en kilometros.
* 
* @author Ricardo Recaredo
*
* @param double $lat1 Latitud inicial
* @param double $long1 Longitud inicial
* @param double $lat2 Latitud final
* @param double $long2 Longitud final
*
* @return double 
*/

function distanciaGeodesica($lat1, $long1, $lat2, $long2){\ldots}
\end{verbatim}
}
\caption{\footnotesize Ejemplo de Comentarios}
\end{algorithm}

\renewcommand{\baselinestretch}{1.5}



\section{Proceso de Realidad Aumentada}
\label{sec:pdra}

\subsection{Especificaci\'on de Funcionalidades}

\newpage


\chapter{Resultados}
\label{chap:resultados}

Luego de haber finalizado el desarrollo y la validación del prototipo, se presentan a continuación los resultados obtenidos:

\begin{itemize}
    \item Se obtuvo una ontología escrita en \textit{OWL} aplicable para modelar cualquier Disciplina de las ciencias o las humanidades a través de la creación de las instancias correspondientes a ella.
    \item Se obtuvo un modelo de conocimiento del área de \textit{Ciencias de la Computación} escrito en \textit{OWL}, basado en la ontología mencionada en el punto anterior.
    \item Se obtuvo una herramienta de carácter prototípico para la administración de las instancias (anotaciones) del modelo de conocimiento mencionado en el punto anterior.
    \item Se obtuvo un lenguaje básico de consultas basado en \textit{Forma Normal Disyuntiva}.
    \item Se obtuvo un prototipo de buscador sobre el modelo de conocimiento del área de \textit{Ciencias de la Computación} que implementa el lenguaje mencionado en el punto anterior.
\end{itemize}

\chapter{Conclusiones y Recomendaciones}
\label{chap:conclusiones}

\section{Conclusiones}
...

\section{Recomendaciones}

...

\pagestyle{plain}

\begin{thebibliography}{20}

    \bibitem{swp} Antoniou, G. y Van Harmelen, F. (2003). \emph{A Semantic Web Primer}. Massachusetts: MIT Press.

    \bibitem{planeta} Cobo, C. y Pardo, H. (2007). \emph{Planeta 2.0: Inteligencia Colectiva o Medios Fast Food}. México DF: Grup de Recerca d'Interaccions Digitals.

    \bibitem{exito} Éxito Explorador (Agosto 31, 2010). \emph{Estadísticas Mundiales de Internet} [Datos en línea]. Disponible: http://www.exitoexportador.com/stats.htm [Consulta: 2010, Noviembre 28].

    \bibitem{gbws} W3C. \emph{Guía Breve de la Web Semántica} [Documento en línea]. Disponible: http://www.w3c.es/divulgacion/guiasbreves/websemantica [Consulta: 2010, Noviembre 28].

    \bibitem{ricardo} Casanova, R. (2010). \emph{El Modelo Web 2.0}. Presentación sobre el modelo de negocios orientado a la Web 2.0.

    \bibitem{rdfp} W3C. \emph{RDF Primmer} [Documento en línea]. Disponible: http://www.w3.org/TR/2004/REC-rdf-primer-20040210/ [Consulta: 2010, Noviembre 28].

    \bibitem{rdf} W3C. \emph{Resource Description Framework (RDF)} [Documento en línea]. Disponible: http://www.w3.org/RDF/ [Consulta: 2010, Noviembre 29].

    \bibitem{rdffaq} GSI. \emph{RDF Schema (RDFS)} [Presentación en línea]. Disponible: www.gsi.dit.upm.es/~gfer/ssii/RDFS.pdf [Consulta: 2010, Noviembre 30].

    \bibitem{owlfaq} W3C. \emph{Preguntas Frecuentes del OWL} [Documento en línea]. Disponible: http://www.w3c.es/Traducciones/es/SW/2005/owlfaq [Consulta: 2010, Noviembre 30].

    \bibitem{owl} W3C. \emph{OWL: Use Cases and Requirements} [Documento en línea]. Disponible: http://www.w3.org/TR/2004/REC-webont-req-20040210/\#onto-def [Consulta: 2010, Diciembre 01].

    \bibitem{sparql} W3C. \emph{SPARQL Query Language for RDF} [Documento en línea]. Disponible: http://www.w3.org/TR/rdf-sparql-query/ [Consulta: 2010, Diciembre 02].

    \bibitem{se} Diaz, S. (s/f). \emph{Sistemas Expertos. Un paso en la simulación del razonamiento humano} [Documento en línea]. Disponible: http://www.monografias.com/trabajos23/sistemas-expertos/sistemas-expertos.shtml [Consulta: 2010, Noviembre 30]

    \bibitem{manifiesto} Beck, K. et al. (2001). \emph{Manifesto for Agile Software Development} [Documento en línea]. Disponible: http://agilemanifesto.org/ [Consulta: 2010, Diciembre 02].

    \bibitem{pressman} Pressman, R. (2010). \emph{Software Engineering: a Practitioner's Approach}. New York: McGraw Hill.

    \bibitem{sommervile} Sommervile, I. (2007). \emph{Software Engineering}. New York: Pearson Education.

    \bibitem{foaf} The Friend of a Friend Project. \emph{FOAF Project} [Documento en línea]. Disponible: http://www.foaf-project.org [Consulta: 2011, Enero 15].

    \bibitem{python} Python Software Foundation. \emph{About Python} [Documento en línea]. Disponible en: http://www.python.org/about [Consulta: 2011, Agosto 08].

    \bibitem{java-joyanes} Joyanes, L. y Zahonero I. (2002). \emph{Programación en Java 2}. Madrid: McGraw Hill.

    \bibitem{jena} Jena Community (s/f). \emph{About Jena} [Documento en línea]. Disponible: http://jena.sourceforge.net/index.html [Consulta: 2011, Agosto 08].

    \bibitem{sesame} Autor desconocido (s/f). \emph{About Sesame} [Documento en línea]. Disponible: http://www.openrdf.org/about.jsp [Consulta: 2011, Agosto 08].

    \bibitem{redland} RedLand Community (s/f). \emph{RedLand FAQ} [Documento en línea]. Disponible: http://librdf.org/FAQS.html [Consulta: 2011, Agosto 08].

    \bibitem{cubicweb} CubicWeb Community (s/f). \emph{The Semantic Web is a Construction Game} [Documento en línea]. Disponible http://www.cubicweb.org/ [Consulta: 2011, Agosto 08].

    \bibitem{virtuoso} Virtuoso Community (s/f). \emph{Virtuoso FAQ} [Documento en línea]. Disponoble: http://virtuoso.openlinksw.com/dataspace/dav/wiki/Main/VOSVirtuoso6FAQ [Consulta: 2011, Agosto 08].

    \bibitem{debian} Debian Community (s/f). \emph{About Debian} [Documento en línea]. Disponible: http://debian.org [Consulta: 2011, Diciembre 20]

    \bibitem{techradar} Autor Desconocido (s/f). \emph{10 Best Linux Distros for 2011} [Documento en línea]. Disponible: http://www.techradar.com/news/software/operating-systems/10-best-linux-distros-for-2011-704584?artc\_pg=2 [Consulta: 2012, Enero 10]

    \bibitem{norvig-russel} Norvig, P. y Russell, S. (2004). \emph{Inteligencia Artificial Un Enfoque Moderno} (2a. ed). Madrid: Prentice Hall.

    \bibitem{aho} Aho, A. et al. (2007). \emph{Compilers Principles, Techniques & Tools} (2a. ed). Boston: Pearson Education.

    \bibitem{lucia} Cardoso, L. (2006). \emph{Sistemas de Base de Datos II, Teoría aplicada para profesores y estudiantes}. Caracas: Publicaciones UCAB.

    \bibitem{papelLuque} Luque, M. et al. (s/f). \emph{Caracterización Automática de Perfiles de Usuarios basados en Consultas Lingüísticas Multigranulares usando un Algoritmo Genético Multiobjetivo}. España.

\bibitem{ontoclean} Guarino, N. y Welty, C. (s/f). \emph{An Overview of OntoClean}. Italia.

\bibitem{kuster} Kúster, W y Rauseo, O. (2009). \emph{Implementar un servicio web que permita crear mashups públicos o privados para ser usados a través de la mensajería de texto celular}. Trabajo Especial de Grado para optar al título de Ingeniero en Informática, Universidad Católica Andrés Bello. Caracas.

\bibitem{ieeeacm} IEEE-CS y ACM (2001). \emph{Computing Curricula 2001 Computer Science} [Documento en línea]. Disponible: http://www.acm.org/education/education/education/curric\_vols/cc2001.pdf [Consulta: 2011, Marzo 15].

\bibitem{acm} IEEE-CS y ACM (2004). \emph{Computer Engineering 2004: Curriculum Guidelines for Undergraduate Degree Programs in Computer Engineering} [Documento en línea]. Disponible: http://www.acm.org/education/education/curric\_vols/CE-Final-Report.pdf [Consulta: 2011, Marzo 15].

\bibitem{ieee} IEEE-CS y ACM (2008). \emph{Computer Science 2008: an Interim Revision of CS 2001} [Documento en línea]. Disponible: http://www.acm.org//education/curricula/ComputerScience2008.pdf [Consulta: 2011 Marzo, 15]

\bibitem{progsemweb} Hebeler, J. et al. (2009). \emph{Semantic Web Programming}. Indiana: Wiley Publishing, Inc.

\end{thebibliography}



% es un cap'itulo sin n'umero.
\chapter*{Lista de abreviaturas}  
\label{chap:abrev}
\addcontentsline{toc}{chapter}{Lista de abreviaturas} 


\begin{center}
\begin{longtable}{r p{13.5cm}} % Se usa longtable porque posiblemente no quepa en una sola p'agina

\textbf{3G} & Del espa\~nol: tercera generaci\'on de transmisi\'on de voz y datos a trav\'es de telefon\'ia m\'ovil\\

\textbf{AVD} & Del ingl'es:\textit{ Android Virtual Devices} , en espa\~nol: Dispositivos Virtuales de Android\\

\textbf{ADT} & Del ingl'es:\textit{ Android Development Tools} , en espa\~nol: Herramientas de Desarrollo de Android\\

\textbf{API} & Del ingl'es: \textit{Application Programming Interface}, en espa\~nol: Interfaz de Programaci'on de Aplicaciones.\\

\textbf{BTS} & Del ingl'es: \textit{Base Transceiver Station}, en espa\~nol: Estaci\'on Base.\\

\textbf{BSC} & Del ingl'es: \textit{Base Station Controller}, en espa\~nol: Controlador de Estaciones Base.\\

\textbf{CCS} & Del ingl'es:\textit{ Cascading Style Sheets}, en espa\~nol: Hojas de Estilo en Cascada.\\

\textbf{CMS} & Del ingl'es: \textit{Content Management System}, en espa\~nol: Sistema Manejador de Contenido.\\

\textbf{EDGE} & Del ingl'es: \textit{Enhanced Data rates for GSM of Evolution}, en espa\~nol: Tasas de Datos Mejoradas para la Evoluci\'on de GSM.\\

\textbf{SDK} & Del Ingl'es: \textit{Software Development Kit}, en espa\~nol: Kit de Desarrollo de Software.\\

\textbf{GNU} & Del ingl'es:  GNU \textit{is Not} Unix, en espa\~nol: Acr'onimo recursivo que significa GNU No es Unix. \\

\textbf{GPL} & Del ingl'es: \textit{General Public License}, en espa\~nol: Licencia P'ublica General. \\

\textbf{GPS} & Del ingl'es: \textit{Global Positioning System}, en espa\~nol: Sistema de Posicionamiento Global. \\

\textbf{GPRS} & Del ingl'es: \textit{General Packet Radio Service}, en espa\~nol: Servicio General de Paquetes V\'ia Radio. \\

\textbf{HTTP} & Del ingl'es: \textit{HyperText Transfer Protocol}, en espa\~nol: Protocolo de Transferencia de Hipertexto. \\

\textbf{IDE} & Del ingl'es:  \textit{integrated development environment}, en espa\~nol: Entorno de desarrollo integrado. \\

\textbf{JDK}  & Del ingl'es: \textit{Java Development Kit}, en espa\~nol:Kit de desarrollo de Java.\\

\textbf{jSON}  & Del ingl'es: \textit{JavaScript Object Notation}, en espa\~nol:Kit Notaci\'on de Objetos de JavaScript.\\

\textbf{OMS} & Del Espa\~nol:  Organizaci\'on Mundial de la Salud.\\

\textbf{RAE} & Del Espa\~nol: Real Academia Espa\~nola. \\

\textbf{RF} & Del espa\~nol: Ruta de Radio-Fecuencia\\

\textbf{TIC} & Del espa\~nol: Tecnolog\'ias de la informaci\'on y la comunicaci\'on\\

\textbf{OS} & Del ingl'es: \textit{Operating System}, en espa\~nol: Sistema Operativo. \\

\textbf{PHP} & Del ingl'es: \textit{PHP Hypertext Pre-processor}.\\

\textbf{POI} & Del ingl'es: \textit{Point of Interest}, en espa\~nol: \textit{Puntos de Inter\'es}.\\

\textbf{XML} & Del ingl'es: \textit{Extensible Markup Language}, en espa\~nol: Lenguaje de Marcado Extensible. \\

\textbf{XP} & Del ingl'es: \textit{Xtreme Programming}, en espa\~nol: Programaci\'on extrema. \\

\end{longtable}\end{center}

\addtocontents{toc}{\vspace{2em}} % Add a gap in the Contents, for aesthetics

\pagenumbering{roman}

\pagestyle{empty}

\vspace*{3.5cm}
\begin{center}
\begin{LARGE}
Aplicaci\'on de apoyo a los procesos de ubicaci\'on de nuevos enlaces corporativos y de selecci\'on de los sitios de transmisi\'on a los que se conectar\'an

(AP\'ENDICES)
\end{LARGE}
\vspace{\fill}
{
\begin{tabular}{p{6.5cm}l}
& \textbf{Ricardo Recaredo}\tabularnewline
 & \textbf{Viviana Trujillo}\tabularnewline
\multicolumn{1}{l}{} & \multicolumn{1}{l}{}\tabularnewline
\multicolumn{1}{l}{}& \multicolumn{1}{l}{\textbf{Jorge Robaina}}\tabularnewline
\multicolumn{1}{l}{} & \multicolumn{1}{l}{}\tabularnewline
\multicolumn{1}{l}{} & \multicolumn{1}{l}{}\tabularnewline
\multicolumn{1}{l}{}& \multicolumn{1}{l}{Caracas, 29 de septiembre de 2010} \tabularnewline
\end{tabular}
}
\end{center}


\pagenumbering{arabic}
\setcounter{page}{1} 

\pagestyle{ruled}

\appendix % Cue to tell LaTeX that the following 'chapters' are Appendices
\clearpage

%Marco Teorico
\include{apendiceG} 		% Appendix GSM
\chapter{M\'etodo de Ponderaci\'on Lineal (scoring)}
\label{AppendixMPL}

\section{Matriz de ponderaci\'on del Score para el m\'etodo de ponderaci\'on lineal}
Para tener una visi\'on mas amplia del escenario de evaluaci\'on y de la comparativa de los Scores sobre los diferentes criterios para cada una de las alternativas, es recomendable elaborar una Matriz de ponderaci\'on de alternativas basada en el peso asignado a los criterios. 

Lo usual es que en la Matriz De Ponderaci\'on en su forma general[\ref{tbl:matrizPonderacion}], en la primera columna se presenten las alternativas a ser evaluadas y en las siguientes columnas los criterios, dejando la primera fila para identificar los respectivos criterios y los rangos de sus pesos y las restantes casillas de la matriz para realizar la valoraci\'on propiamente dicha, y se conserva la \'ultima columna para completar la evaluaci\'on de cada alternativa, sumando los puntos acumulados por la misma, en su respectiva fila.

\renewcommand{\baselinestretch}{1}

\nuevatabla{matrizPonderacion}
{\footnotesize Matriz De Ponderaci\'on en su forma m\'as general}
{\tiny Fuente: Matrices de Ponderaci\'on para la Evaluaci\'on de Proveedores. 2007}
{p{2cm}p{1.5cm}p{1.5cm}p{0.5cm}p{2cm}p{1.5cm}p{1cm}}
{
\multicolumn{1}{>{\columncolor{vivinicered}}c}{} &
\multicolumn{1}{>{\columncolor{vivinicered}}c}{\footnotesize \textcolor{white}{\textbf{Criterio 1}}} &
\multicolumn{1}{>{\columncolor{vivinicered}}c}{\footnotesize \textcolor{white}{\textbf{Criterio 2}}}& 
\multicolumn{1}{>{\columncolor{vivinicered}}c}{}  & 
\multicolumn{1}{>{\columncolor{vivinicered}}c}{\footnotesize \textcolor{white}{\textbf{Criterio n-1}}} & 
\multicolumn{1}{>{\columncolor{vivinicered}}c}{\footnotesize \textcolor{white}{\textbf{Criterio n }}} & 
\multicolumn{1}{>{\columncolor{vivinicered}}c}{}\\
 
\multicolumn{1}{>{\columncolor{vivinicered}}c}{\footnotesize \textcolor{white}{\textbf{Alternativas}}} & 
\multicolumn{1}{>{\columncolor{vivinicered}}c}{\footnotesize \textcolor{white}{\textbf{Peso x a X}}} & 
\multicolumn{1}{>{\columncolor{vivinicered}}c}{\footnotesize \textcolor{white}{\textbf{Peso y a Y}}} & 
\multicolumn{1}{>{\columncolor{vivinicered}}c}{\footnotesize \textcolor{white}{\textbf{\ldots{}}}} & 
\multicolumn{1}{>{\columncolor{vivinicered}}c}{\footnotesize \textcolor{white}{\textbf{Peso w a W}}}  & 
\multicolumn{1}{>{\columncolor{vivinicered}}c}{\footnotesize \textcolor{white}{\textbf{Peso z a Z}}} & 
\multicolumn{1}{>{\columncolor{vivinicered}}c}{\footnotesize \textcolor{white}{\textbf{Total}}} \\ \hline

\multicolumn{1}{c}{\footnotesize Alternativa 1} &
\multicolumn{1}{c}{\footnotesize P1,1} & 
\multicolumn{1}{c}{\footnotesize P1,2} & 
\multicolumn{1}{c}{\footnotesize \ldots{}} & 
\multicolumn{1}{c}{\footnotesize P1,n-1}  & 
\multicolumn{1}{c}{\footnotesize P1,n} & 
\multicolumn{1}{c}{\footnotesize Total 1} \\ 

\multicolumn{1}{>{\columncolor{cap}}c}{\footnotesize Alternativa 2} & 
\multicolumn{1}{>{\columncolor{cap}}c}{\footnotesize P2,1} & 
\multicolumn{1}{>{\columncolor{cap}}c}{\footnotesize P2,2} & 
\multicolumn{1}{>{\columncolor{cap}}c}{\footnotesize \ldots{}} & 
\multicolumn{1}{>{\columncolor{cap}}c}{\footnotesize p2,n-1}  & 
\multicolumn{1}{>{\columncolor{cap}}c}{\footnotesize P2,n} & 
\multicolumn{1}{>{\columncolor{cap}}c}{\footnotesize Total 2}\\

\multicolumn{1}{c}{\footnotesize \ldots{}} &
\multicolumn{1}{c}{\footnotesize \ldots{}} & 
\multicolumn{1}{c}{\footnotesize \ldots{}} & 
\multicolumn{1}{c}{\footnotesize \ldots{}} & 
\multicolumn{1}{c}{\footnotesize \ldots{}} & 
\multicolumn{1}{c}{\footnotesize \ldots{}} & 
\multicolumn{1}{c}{\footnotesize \ldots{}}\\

\multicolumn{1}{>{\columncolor{cap}}c}{\footnotesize Alternativa m-1} &  
\multicolumn{1}{>{\columncolor{cap}}c}{\footnotesize Pm-1,1} & 
\multicolumn{1}{>{\columncolor{cap}}c}{\footnotesize Pm-1,2} & 
\multicolumn{1}{>{\columncolor{cap}}c}{\footnotesize \ldots{}} & 
\multicolumn{1}{>{\columncolor{cap}}c}{\footnotesize Pm-1,n-1} & 
\multicolumn{1}{>{\columncolor{cap}}c}{\footnotesize Pm-1,n} & 
\multicolumn{1}{>{\columncolor{cap}}c}{\footnotesize Total m-1}\\

\multicolumn{1}{c}{\footnotesize Alternativa m} & 
\multicolumn{1}{c}{\footnotesize Pm,1} & 
\multicolumn{1}{c}{\footnotesize Pm,2} & 
\multicolumn{1}{c}{\footnotesize \ldots{}} & 
\multicolumn{1}{c}{\footnotesize Pm,n-1} & 
\multicolumn{1}{c}{\footnotesize Pm,n} & 
\multicolumn{1}{c}{\footnotesize Total m}\\ \hline
}

\renewcommand{\baselinestretch}{1.5}

En la forma m\'as usual los rangos son todos iguales, es decir x = y = \ldots{} = w = z, en general igual a cero (0) o uno (1) como valor inicial y X = Y = \ldots{} = W = Z = a un valor fijo, generalmente diez (10), veinte (20), cincuenta (50), cien (100) o mil (1000), de acuerdo a lo dif\'icil que puede ser diferenciar entre una alternativa y otra. La otra variante, comnmente usada da valores diferentes a los rangos de cada criterio, en este caso los valores de x, y, w, z, no necesariamente son ceros o unos, y los valores de X, Y, W, Z, suelen ser diferentes, entendi\'endose que entre mayor sea este \'ultimo valor, mayor peso se le desea dar a ese criterio y entre mayor sea la diferencia del mayor valor de un criterio menos el menor valor del mismo (X \textendash{} x, \ldots{}, Z \textendash{} z), se desea ser m\'as diferenciador en cuanto a este criterio. 

 		% Appendix Metodo de ponderacion lineal

\addtocontents{toc}{\vspace{2em}}  

%\backmatter

\end{document}