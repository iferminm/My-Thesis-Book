% Appendix G
\chapter{GSM}
\label{AppendixG}

\section{Arquitectura del Sistema GSM}

La arquitectura del sistema GSM se compone de cuatro bloques o subsistemas que engloban el conjunto de elementos de la jerarqu\'ia del sistema. Cada uno de estos subsistemas desempe\~na funciones espec\'ificas para, en su conjunto, ofrecer el servicio de telefon\'ia m\'ovil al usuario final. Los cuatro subsistemas son:

\begin{itemize}
\item La estaci\'on m\'ovil (MS)
\item El subsistema de estaci\'on base (BSS)
\item El subsistema de conmutaci\'on y de red (NSS)
\item El subsistema de operaci\'on y mantenimiento (MNS).
\end{itemize}

\pngfig[1]{arqGsm}{Arquitectura del Sistema GSM}{Fuente: Principios de Comunicaciones M\'oviles}

\subsection{La Estaci\'on M\'ovil (MS, Mobile Station)}
La estaci\'on m\'ovil es el punto de entrada a la red m\'ovil inal\'ambrica. Es el equipo f\'isico usado por el usuario GSM para acceder a los servicios proporcionados por la red.

\subsection{Subsistema de la Estaci\'on Base (BSS, Base Station Controller)}
\label{sec:bss}
El subsistema de estaci\'on base (BSS) agrupa la infraestructura espec\'ifica de los aspectos radio para el sistema GSM. Este subsistema se compone de estaciones base (BTS) conectadas a una estaci\'on controladora de BTSs, la denominada BSC.

\subsubsection{Estaci\'on de Transmisi\'on-Recepci\'on Base (BTS, Base Transceiver Station)}
\label{sec:bts}
La unidad BTS es la parte del subsistema BSS que dispone de los dispositivos para la transmisi\'on y recepci\'on radio, incluyendo las antenas. Realiza las tareas de conformaci\'on de la se\~nal a transmitir v\'ia radio y de recuperaci\'on de la se\~nal radio en recepci\'on, adem\'as de realizar el procesado digital de la se\~nal, codificaci\'on de canal, entrelazado, etc. Normalmente se ubica en el centro geogr\'afico de la celda y la potencia m\'axima emitida determina el tama\~no absoluto de la celda.

\subsubsection{Controlador de Estaciones Base (BSC, Base Station Controller)}
\label{sec:bsc}
La unidad BSC se encarga de administrar los recursos radio mediante el comando remoto de las BTS.
Su funci\'on consiste b\'asicamente en la asignaci\'on y liberaci\'on de los canales radio, as\'i como en la gesti\'on del traspaso de llamada cuando \'este se produce entre estaciones base dependientes de la misma BSC. Tambi\'en se encarga del cifrado de la comunicaci\'on y de la ejecuci\'on de los algoritmos de transmisi\'on discontinua, mediante la detecci\'on de los per\'iodos de actividad y silencio en las comunicaciones.

\subsubsection{Unidad de Transcodificaci\'on (TRAU-Transcoding Rate and Adaptation Unit)}
Se encarga de comprimir la informaci\'on en el interfaz a\'ereo cuando se hace necesario. La TRAU forma parte del subsistema BSS. Permite que tasas de datos GSM (8,16,32 Kbps).

\subsection{Subsistema de red y conmutaci\'on (NSS, Network Switching System)}
Realiza las funciones de conmutaci\'on y encaminamiento de las llamadas en el sistema GSM, adem\'as de la gesti\'on de las bases de datos con la informaci\'on relativa a todos los abonados al servicio. El NSS se encarga de establecer la comunicaci\'on entre usuarios m\'oviles mediante la conmutaci\'on interna de red de un operador o entre usuarios del sistema GSM y usuarios de otras redes de telefon\'ia, ya sea de telefon\'ia fija o de telefon\'ia m\'ovil de otros operadores.

Dentro del subsistema NSS las funciones de conmutaci\'on las realizan las centrales de conmutaci\'on).

Dentro del subsistema NSS las funciones de conmutaci\'on las realizan las centrales de conmutaci\'on (MSC y GMSC). 

La unidad MSC es el elemento de conmutaci\'on interno de una red GSM mientras que la unidad GMSC (Gateway Mobile Switching Center) es el elemento de interconexi\'on con otras redes. La gesti\'on de las bases de datos la realizan el registro central de abonados (HLR) y el registro de posiciones visitante (VLR).

El registro general de abonados (HLR-Home Location Register). Es una base de datos que contiene y administra la informaci\'on de los abonados, mantiene y actualiza la posici\'on del m\'ovil y la informaci\'on de su perfil de servicio.

El VLR es el registro de posiciones visitante y contiene informaci\'on temporal de los m\'oviles que est\'an localizados en un \'area geogr\'afica concreta. La informaci\'on del VLR es una repetici\'on sesgada de la informaci\'on de un abonado contenida en el HLR complementada con informaci\'on temporal relativa a la ubicaci\'on en ese momento del terminal m\'ovil.

\subsection{Subsistema de operaci\'on y mantenimiento (OSS)}
Las acciones de operaci\'on y mantenimiento se llevan a cabo con el fin de conseguir el buen funcionamiento del sistema GSM en su conjunto, ya sea solucionando los problemas y fallos que aparezcan o monitorizando y mejorando la configuraci\'on de los equipos para un mayor rendimiento.
La gesti\'on y mantenimiento se puede realizar de forma local o remota. Para redes de tama\~no considerable, debido a la complejidad de los sistemas de telecomunicaci\'on, la gesti\'on remota se convierte en una necesidad.
